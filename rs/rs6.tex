\documentclass[12pt]{article} \usepackage{color} \usepackage{amsmath} \usepackage{amssymb}
\usepackage{graphicx} \usepackage{times} \usepackage{listings}

\usepackage[total={5.7in,8.89in},bindingoffset=0.01in,top=1.25in,
left=1.4in, includefoot]{geometry}
\usepackage[dvips,color,line,arrow,graph,frame,matrix]{xy}

\definecolor{listingbg}{rgb}{0.98,0.98,0.98}
\lstset{ basicstyle=\ttfamily, backgroundcolor=\color{listingbg},
         frame=tb }

\begin{document}
\begin{center}
{\large Review Set 6} 
\end{center}

This Review Set asks you to prepare written answers to questions on
local and global optimizations. Each of the questions has a short
answer. You may discuss this Review Set with other students and work
on the problems together. 

\begin{enumerate}
\item{ Consider the rules for the constant propagation algorithm discussed in class:
\begin{equation*}
\begin{array}{cl}
(1) & C_{in}(x, s) = \#{} \quad \Rightarrow \quad C_{out}(x, s) = \#{} \\
(2) & C_{in}(x, x \leftarrow c) \quad \Rightarrow \quad C_{out}(x, x \leftarrow c) = c \quad (c \text{ is a constant})\\
(3) & C_{out}(x, x \leftarrow f(\ldots)) = * \\
(4) & E(x) \neq E(y) \quad \Rightarrow \quad C_{out}(x, y \leftarrow \ldots) = C_{in}(x, y \leftarrow \ldots) \\
(5) & C_{in}(x,s) = lub \{ C_{out}(x, p) \ | \ p \text{ is a predecessor of } s \}
\end{array}
\end{equation*}

Note that these rules are not necessarily applied in order---the
numbers are just there for ease of referencing. See the
``Dataflow Analysis, Global Optimizations'' lecture notes for the actual steps
of the algorithm.

\begin{enumerate}
\item Give a concise English description for each of rules 1--4.
\item Note that rules 1--4 define $C_{out}$ in terms of $C_{in}$. Rule 5,
on the other hand, defines $C_{in}$ based on the $C_{out}$ values of
all predecessor statements. Give two distinct examples that show some set of
predecessor $C_{out}$'s and the resulting $C_{in}$ computation.
\item Briefly explain why the algorithm (as described in the lecture notes) is
guaranteed to terminate.
\item Note that, in rule 4, we set $y \leftarrow \ldots$ (where
``$\ldots$'' is some expression $e$). Why is it safe to assume that the
evaluation of $e$ does not change the value of $x$?
\end{enumerate}
}

\newpage  

\item Consider the following fragment of intermediate code:
\begin{lstlisting}
    START
    if a = 2 goto L3
L0: b := 2
L1: d := a / 2
    c := a % b
    if c = 0 goto L2
    if b >= d goto L3
    b := b + 1
    goto L1
L2: a := a + 1
    goto L0
L3: END

\end{lstlisting}
\begin{enumerate}
\item Divide this code into basic blocks; there should be at least
6. Assume that \texttt{START} and \texttt{END} are placeholder
instructions (i.e. they don't do anything).
\item Draw a control-flow graph for this program, using your answer to
(a). Place each basic block in a single node.
\item Annotate your control-flow graph with the set of variables that
are live before and after each statement. {\bf Assume that only
\texttt{a} is live at the entry to \texttt{L3}}.
\item Describe concisely what this program does if the value of
  \texttt{a} is the only output.
 \end{enumerate}
\end{enumerate}
\end{document} 
