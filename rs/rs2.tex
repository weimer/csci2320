\documentclass[12pt]{article}
\usepackage{graphicx}

\begin{document}

\begin{center}
{\large Review Set 2} 
\end{center}

This Review Set asks you to prepare written answers to questions on
context-free grammars. Each of the questions has a short answer. You may
discuss this Review Set with other students and work on the problems
together.  

\begin{enumerate}

% #1
\item Let $L_1$ be the language consisting of all non-empty palindromes
over the alphabet $\Sigma = \{a, b\}$.  That is, $L_1$ consists of all
sequences of $a$'s and $b$'s that read the same forward or backward.  For
example, $aba \in L_1$ and and $bb \in L_1$ and $aabbbaa \in L_1$, but $abb
\notin L_1$. 

Let $L_2$ be the language over $\Sigma = \{a, b\}$ denotated by the regular
expression $a(a|b)*$. 

The language $L_3 = L_1 \cap L_2$ is context-free. A string $s$ is in $L_3$
if $s \in L_1$ and $s \in L_2$. Write a context-free grammar for the
language $L_3$. 

{\bf Optional Thing To Think About:} Is the intersection of a context-free
language and a regular language always context-free? 

% #2
\item Consider the following grammar:

\begin{eqnarray*}
S & \rightarrow & aSb \\
S & \rightarrow &  Sb\\
S & \rightarrow & \epsilon \\
\end{eqnarray*}

\begin{enumerate}

\item Give a one-sentence description of the language generated by this
grammar.

\item Show that this grammar is ambiguous by giving a single string that
can be parsed in two different ways.  Draw both parse trees.

\item Give an unambiguous grammar that accepts the same language as the
grammar above.

\end{enumerate}

% #3
\item Using the context-free grammar for Cool given in the Cool Reference
Manual, draw a parse tree for the following expression.

\begin{verbatim}
    while not (x <- z <- 0) loop 
      y <- z + 2 * x + 1 
    pool  
\end{verbatim}

Note that the context-free grammar by itself is ambiguous, so you will
need to refer to the precedence and associativity rules to get
the correct tree. 

% #4
\item Give an example of a grammar that is $LL(3)$ but not $LL(2)$.

\end{enumerate}

\end{document}
